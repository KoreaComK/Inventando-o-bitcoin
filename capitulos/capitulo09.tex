%\part{Passado, Presente e Futuro}
\chapter{O que vem depois}
\label{ch:capitulo9}
\section*{O Bitcoin é o Orkut das Cripto?}
Por que eu escolhi escrever um livro sobre Bitcoin quando eu poderia ter escrito um sobre o ecossistema cripto como um todo?
Não existe milhares de outras moedas? O que faz o Bitcoin ser tão especial, Além que foi a primeira criptomoeda descentralizada.
Ela não é mais lenta que seus competidores?

Isso é o que muitas pessoas novas ao bitcoin perguntam. 
Depois de entender o básico de como Bitcoin funciona, a próxima pergunta logica tende a ser: "Tecnológica blockchain parece interessante. Como sabemos que uma versão melhor não vai surgir e transformar meu Bitcoin no Orkut das cripto?"

O fosso é uma vantagem competitiva que um negocio construí para impedir a entrada de novos concorrente.
Para o Orkut esse fosso era uma enorme base de usuários que mantinham relações amigáveis. 
Pessoas não utilizariam um serviço competidor que seus amigos também não estivessem la.
Mas tanto um fosso quanto um grafo social bem conectado não foram o suficiente para impedir o Facebook de comer o almoço do Orkut ao longo de alguns poucos anos.

O fosso do Bitcoin é muito mais muito maior que o do Orkut. Para entender isso precisamos examinar o que um competidor precisaria fazer para deslocar o Bitcoin

\section*{Seja o dinheiro mais comercializado e liquido}

A primeira coisa a entender é que a comparação com Orkut versus Facebook é ruim porque você pode ter uma conta no Orkut e no Facebook ao mesmo tempo sem custo. 
Isso na verdade é o que muitas pessoas fizeram durante a fase de transição de uma rede social a outra.
Quando a massa critica de pessoas haviam migrado para o Facebook, as pessoas pararam de usar o Orkut.

Não é assim que dinheiro funciona, entretanto. Se você tem o valor de um dólar em bitcoin, isso é um dólar de valor que você não pode ter em nenhuma outra moeda.
você esta tomando a decisão de maneira consciente de vender uma moeda por outra.
você não pode guardar o mesmo valor em ambas as moedas ao mesmo tempo. 
Agora se pergunte: Por que você iria segurar qualquer coisa se não a moeda mais liquida e mais amplamente aceita?
A resposta é apenas especulação. 
Se você não consegue mudar a economia inteira ao seu redor para também guardar essa moeda, então não existe maneira dessa moeda se tornar dominante.

A liquidez do Bitcoin é muito além de qualquer um dos seus competidores. 
No dia de hoje, o valor de mercado do Bitcoin é de \$160B de acordo com \url{https://messari.io/screener}. O segundo maior competidor, o Etherium, tem apenas \$30B de valor de mercado.
Isso nem é uma mensura da verdadeira liquidez observando quanto você consegue efetivamente vender antes do preço começar a decair significativamente.

Liquidez é uma bola de neve.
Segurar o dinheiro mais liquido significa que mais gente quer ele, que aumenta ainda mais a liquidez dele. 
Retendo qualquer outra coisa que não seja o dinheiro mais liquido, você esta ativamente punindo a se mesmo enquanto espera todas as outras pessoas fazerem o mesmo. 
Os incentivos econômicos não se alinham em favor da liquidez do concorrente em uma noite.

\section*{Demonstração de \$100B+ de valor seguro por dez anos}

Através das circunstancias, foi permitido o Bitcoin crescer de um experimento geek na internet que ninguém se importava, para, comprar uma pizza por 10,000 bitcoins, para, um preço pico de \$20K USD por bitcoin. Ele fez tudo isso de maneira relativamente quieta, sem ninguém enchendo o saco. 
Durante esse período, ele desenvolveu um sistema imunológico de nível mundial, devido a anos de ataques e cresceu para a rede de maior taxa de hash do mundo. 
Em dez anos, assegurando mais que 100 bilhões de dólares, e se tornou impossível de hackear.

É quase impossível lançar uma nova criptomoeda quietamente hoje.
Hoje a coisa está em alta, e todo mundo está ligado neste mercado.
Vamos analisar uma blockchain alternativa, o EOS, que \href{https://coinmarketcap.com/currencies/eos/}{\textcolor{red}{valeu aproximadamente 10 bilhões de dólares no seu lançamento}} e hoje vale menos da metade. 
Ele travou dois dias apos seu lançamento devido a uns bugs no seu código. 
Esses bugs foram atualizado dentro de horas com mínima supervisão ou revisão. 
você vai colocar \$ 100B de valor numa rede assim? 
Talvez o EOS ainda esteja por aqui em uns 10 anos mas, talvez nessa época, O Bitcoin vai ter 20 anos de idade, e segurará trilhões em valor.

\section*{Congele ataque vindo taxa de hash}

Visto que das milhares das moedas la fora estão usando uma duzia de algoritmos de hash diferentes, quais quer novas moedas que surjam estão sobre constante ameaça de ataques de 51\% pela taxa de hash já disponível. \href{https://fortune.com/2018/05/29/bitcoin-gold-hack/}{\textcolor{red}{Isso já aconteceu com o Bitcoin Gold}} e diversas \href{https://www.coindesk.com/markets/2018/06/08/blockchains-once-feared-51-attack-is-now-becoming-regular/}{\textcolor{red}{outras moedas}}.

Qualquer novo competidor precisa sobreviver ataque do poder de hash já existe ou usar um algoritmo que não possui nenhuma ASIC. Se não possui nenhuma ASIC então o sistema vai ser facilmente atacado, utilizando um serviço de aluguel de GPU, já amplamente disponíveis. 
Ele também não pode começar assegurando um valor alto, como o EOS fez um dia, que é imprudente e uma boa maneira de ser obrigado a fazer atualizações centralizadas. 
Então eles não podem se financiar, então a única maneira que sobra é um lançamento justo similar ao bitcoin e crescer de valor lentamente para que eles possam desenvolver seu sistema se segurança proporcionalmente. Porem se eles crescerem muito lentamente eles não vão alcançar o numero de usuários e liquidez do Bitcoin devido ao decorrer do tempo.

\section*{Ser altamente descentralizado}

A grande do modelo de segurança do bitcoin vem do seu alto grau de descentralização.
Isso significa que o protocolo é difícil de ser modificado e por consequência pode ser confiado a honrar as propriedades escritas em seu código(Oferta limitada,etc). Essa propriedade foi testada quando um numero alto de empresários e mineradores se juntaram e queriam mudar o tamanho do bloco para conduzir o protocolo em uma dada direção\footnote{Leia mais sobre o tal fork chamado de Segwit2x que foi planejado através de acordos obscuros e consequentemente abortado aqui:\url{https://bitcoinmagazine.com/technical/now-segwit2x-hard-fork-has-really-failed-activate}}. Esse fork foi rejeitado pelos usuários e falhou espetacularmente.

Um competidor que é altamente descentralizado basicamente elimina qualquer empresa ou equipe que são formadas por pessoas conhecidas visto que isso cria um ponto central de falha e coerção. também exclui qualquer moeda que queria "sair quebrando tudo", pois só podes fazer isso quando es centralizado.
Qualquer competidor ou esta indo rápido demais e fica centralizado, ou esta se movendo muito lentamente e nunca vai alcançar.


\section*{Atrair os melhores desenvolvedores do mundo}

Muito semelhante a Linux criou que criou um redemoinho de atividade que impediu o surgimento de outros sistemas *Nix de competir, O bitcoin também. Todo dia a comunidade cresce e novas empresas são montadas em cima do Bitcoin, oferecendo serviços.
Um competidor precisa roubar uma parcela das mentes desenvolvedoras de um núcleo exponencialmente crescente, que inclui duzias de empresas, programas educacionais e conferencias.

\section*{Cresça uma rede financeira global}

Bitcoin é apoiado por \href{https://coinmarketcap.com/currencies/bitcoin/}{\textcolor{red}{centenas de exchanges mundo a fora}}, mercados futuros e outros produtos financeiros derivativos em lugares de grande capital como o \textit{Chicago Mercantile Exchange}, centenas de fundos financeiros e painéis de trading, e ainda uma rede de pessoas que \href{https://www.forbes.com/sites/realspin/2017/02/03/why-venezuelas-currency-crisis-is-a-case-study-for-bitcoin/#4671a1d719b2}{\textcolor{red}{já usam o bitcoin como alternativa a moedas fracassadas como o bolívar venezuelano}}. 
Todas essas coisas precisaram ser construídas para um competidor do Bitcoin deslocar ele.

Instituição como \textit{Chicago Mercantile Exchange} não vão listar cada um dos novos competidores sem que tenha toneladas de volume em exchanges apoiando o competidor. você precisaria convencer negócios a aceitarem esse competidor no lugar do Bitcoin. Um competidor esse que provavelmente é menos seguro, menos liquido, possui uma equipe de desenvolvedores menos competentes e por definição menos adoção mundial. 
É uma subida bastante ingrime a trilhar.

\section*{Seja dinheiro mais forte}

Existe um \href{https://fintechnologynews.com/neither-fast-nor-cheap-choosing-bitcoin-is-foolish-says-nanopay/}{\textcolor{red}{grotesco equivoco que o Bitcoin é a maneira mais rápida e barata de se enviar dinheiro}}. 
Isso claramente não pode ser baseado em suas propriedades fundamentais que envolvem um Livro-razão replicável em escala mundial. 
Entretanto, principal caso de uso já demonstrada é ser dinheiro forte resistente a censura, esta crescendo.

Qualquer outra como, como fazer transferências remetentes mais baratas são cerejas em cima do bolo. A maioria dos tais competidores do Bitcoin ainda pensam que precisam solucionar o caso de uso para pagamentos rápidos, que já foi solucionado por dezenas de empresas centralizadas mudo a fora, e solucionado relativamente bem. Além que também já esta sendo solucionado pelo crescimento acelerado da rede lightning em cima da rede Bitcoin.

Competir na arena de dinheiro forte requer um comprometimento surreal a descentralização e propriedades que são verdadeiramente difíceis de serem modificadas e atacadas. Infelizmente outras moedas não conseguem competir nessa arena visto que na realidade elas foram construídas tipicamente por equipes centralizadas visando o lucro, e não como um bom acidente de um ecossistema lentamente crescente construído por cypherpunks.


%%%%%%%%%%%%%%%%%%%%%%%%%%%%%%%%% ESSA SEçÂO INTEIRA ABAIXO JA ESTA NO CAPITULO 5&&&%%%%%%%%%%%%%%%
%\section{compressao}
% Agora, munidos da compreensão da rede Bitcoin como um todo, podemos examinar alguns comportamentos interessantes que surgiram nos últimos dez anos do sistema.

% \paragraph{O Livre Mercado}
% \paragraph{}

% Mencionamos brevemente as taxas de transação no Capítulo 5 ao discutir a mineração, mas elas merecem sua própria seção. Uma vez que a programação de emissão de Bitcoin consiste em halvings acontecendo a cada quatro anos, até que a Recompensa de Bloco seja totalmente eliminada e o Bitcoin entre em um estado de cunhagem zero até o fim dos tempos, ainda precisamos de uma forma de incentivar os mineradores a continuarem protegendo a rede .

% As taxas são determinadas por um sistema de livre mercado, no qual os usuários pagam por espaço escasso em um bloco. Os usuários que enviam transações indicam quanta taxa estão dispostos a pagar aos mineradores, e eles podem ou não incluir as transações que são informadas, dependendo do quanto irão ganhar. Quando há poucas transações esperando para entrar no próximo bloco, as taxas tendem a ser muito baixas, pois não há competição. À medida que o espaço do bloco é preenchido, os usuários estão dispostos a pagar taxas mais altas para que suas transações sejam confirmadas rapidamente (no próximo bloco). Aqueles que não querem pagar, podem sempre definir taxas baixas e esperar mais para serem minerados em um momento com baixa demanda, quando o espaço do bloco estiver mais disponível.

% Ao contrário dos sistemas financeiros tradicionais, onde as taxas tendem a se basear em uma porcentagem do valor que está sendo transferido, no Bitcoin o valor transferido não tem relação com as taxas. Tornamos as taxas proporcionais ao recurso escasso que consomem: espaço em bloco. Portanto, as taxas são medidas em satoshis por byte (bytes são 8 bits, basicamente apenas uma medida de quantos dados há em sua transação). Assim, uma transação que envia um milhão de bitcoins de um endereço para outro pode ser mais barata do que uma que consolida 1 bitcoin espalhado por 10 contas, porque o último requer mais espaço de bloco.

% No passado, houve períodos em que o Bitcoin tinha uma demanda muito alta, como o que aconteceu no final de 2017, onde as taxas se tornaram extremamente altas. Desde então, alguns novos recursos foram implementados para reduzir a pressão sobre as taxas na rede.

% Um deles é chamado de Segregated Witness (ou Testemunha Segregada), que reorganizou como os dados do bloco são representados separando as assinaturas digitais dos dados da transação, criando mais espaço para esses dados. As transações que tiram proveito desta atualização podem usar mais do que o 1 MB original do espaço do bloco por meio de alguns truques inteligentes que estão além do escopo deste livro.

% O outro alívio para as taxas veio através do batching: As exchanges e outros participantes de alto volume no ecossistema começaram a combinar transações de bitcoin para vários usuários em uma transação. Ao contrário de um pagamento tradicional em seu banco ou PayPal que é feito de uma pessoa para outra, uma transação de Bitcoin pode combinar um grande número de entradas e produzir um grande número de saídas. Assim, uma exchange que precisa enviar bitcoin para saque para 100 pessoas pode fazê-lo em uma única transação. Este é um uso muito mais eficiente do espaço do bloco, transformando o que é ostensivamente apenas um punhado de transações de bitcoin por segundo em milhares de pagamentos por segundo.

% A Segregated Witness e o batching já fizeram um trabalho muito bom na redução da demanda por espaço em bloco. Outras melhorias estão em andamento para tornar o uso do espaço do bloco mais eficiente. No entanto, chegará um momento em que as taxas de Bitcoin ficarão altas novamente, à medida que os blocos ficarem cada vez mais cheios devido à demanda.
 
\section*{Desenvolvimentos Futuros no Bitcoin}
%\paragraph{}

Neste ponto, já passamos por toda a questão de \textit{inventar o Bitcoin} e cobrimos como a rede evoluiu ao longo do tempo. Agora olhamos para o futuro e cobrimos algumas das melhorias de curto prazo que virão para o Bitcoin.

%
%Ao contrário de uma moeda tradicional, que é algo que é impresso e usado, 
O Bitcoin é uma camada de dinheiro programável sobre a qual podemos construir muitos serviços.
Este é um conceito totalmente novo e estamos apenas começando a ter conhecimento do que é possível ser feito.

\subsection*{Lightning Network}
%\paragraph{}

Como discutimos acima, o Bitcoin teve problemas com taxas altas à medida que o espaço em bloco se tornou cada vez mais procurado.
Hoje, o Bitcoin é capaz de apenas cerca de 3 a 7 transações por segundo com base no número de transações que cabem em um bloco.
Lembre-se que cada transação pode, na verdade, ser um pagamento para centenas de pessoas por lote. 
Ainda assim, não tem a capacidade suficiente para se tornar uma rede global de pagamentos.

Uma solução ingênua pode ser aumentar o tamanho do bloco, e de fato várias moedas concorrentes, incluindo o Bitcoin Cash, tentaram essa abordagem.
O Bitcoin não segue esse caminho porque aumentar o tamanho do bloco impactaria negativamente as características de descentralização, como o número de nodes e a dispersão geográfica.
Mesmo que um aumento no tamanho do bloco fosse possível devido a melhorias no hardware, há também o problema de que a natureza descentralizada do Bitcoin significa que um hard fork que tenta mudar o tamanho do bloco causaria muitos problemas, e provavelmente ocorreria outra divisão da blockchain, criando assim, uma moeda diferente.

Um aumento no tamanho do bloco também não resolveria o problema de tornar o Bitcoin adequado como um sistema de pagamento mundial - ele simplesmente não seria tão escalável. É aqui que entra a Lightning Network: Outro protocolo e conjunto de implementações de software que criam transações offchain de Bitcoins.
The Lightning Network pode ser o assunto de todo um livro, mas vamos discuti-la brevemente.

A ideia da Lightning é que nem todas as transações precisam ser registradas na blockchain.
Por exemplo, se você e eu estamos em um bar comprando bebidas, podemos abrir uma conta no bar e resolver no final da noite. 
Realmente não faz sentido cobrarmos de nosso cartão de crédito por cada bebida, pois é uma perda de tempo.
Com o Bitcoin, usar a energia equivalente à de um país inteiro ao confirmar a compra de um café ou cerveja e ter essa compra registrada o tempo todo em milhares de computadores em todo o mundo não é escalonável nem particularmente bom para a privacidade.

A Lightning Network, se for bem-sucedida, melhorará muitas das desvantagens do Bitcoin:

\begin{enumerate}
\item Transferência de transações virtualmente ilimitada. Centenas de milhares de micro transações poderiam ser realizadas usando a blockchain Bitcoin uma vez, como liquidação final;
\item Confirmações instantâneas; não há necessidade de esperar que os blocos sejam minerados;
\item Taxas de transação de menos de um centavo adequadas para micro pagamentos, como pagar um centavo para ler um blog;
\item Maior privacidade. Apenas as partes que participam da transação precisam saber sobre ela, ao contrário de uma transação em rede que é transmitida para o mundo inteiro.
\end{enumerate}

A Lightning usa o conceito de canais de pagamento, que são transações reais de Bitcoin na blockchain que bloqueiam uma certa quantidade de Bitcoin e o tornam disponível na Lightning Network para transferência instantânea e quase gratuita.
A Lightning Network está nos estágios iniciais, mas já se mostra promissora.
Você pode verificar o site \textcolor{red}{\url{https://yalls.org/}} que usa micro pagamentos baseados na Lightning para disponibilizar a leitura de artigos.

\subsection*{Bitcoin no Espaço}
%\paragraph{}

O Bitcoin faz um excelente trabalho de ser resistente à censura, pois é resistente ao confisco (você pode carregá-lo em sua cabeça) e resistente à censura de transferência, uma vez que requer apenas um minerador honesto na rede para garantir suas transações (e você pode minerar você mesmo).

No entanto, sendo o Bitcoin transmitido pela Internet, é suscetível de censura em nível de rede. Os regimes autoritários que querem reprimir a atividade podem tentar bloquear o tráfego de Bitcoin que entra e sai de seu país.

O Blockstream Satellite é o primeiro esforço para contornar a censura de rede em nível estadual, bem como alcançar áreas remotas que podem não ter conexões com a Internet. 
Este satélite permite que qualquer pessoa com uma antena parabólica e equipamento relativamente barato conecte e baixe a blockchain do Bitcoin, com comunicação bidirecional em breve. 
Agora também existem esforços como o TxTenna para construir redes fora da rede elétrica. 
Quando acoplado a uma conexão via satélite, esse tipo de configuração seria quase imparável\footnote{Nota do tradutor: Atualmente um grupo de brasileiros usou ondas de rádio para colocar uma transação na rede usando a Lua como ponto de reflexo. Veja mais em \url{https://livecoins.com.br/brasileiros-enviam-bitcoin-a-lua-na-frente-de-elon-musk/}}.


\section*{Pesquisa Futura}
%\label{ch:capitulo10}

Então é isso. 
Você passou pelo exercício de \textit{Inventar o Bitcoin} e, com sorte, emergiu do outro lado do espelho, pronto para explorar mais sobre o assunto.
Onde você conseguirá mais informações? Aqui estão alguns recursos para ajudá-lo a explorar a toca do coelho:

Para saber mais sobre a economia por trás do Bitcoin:

\begin{itemize}
\item \href{https://amzn.to/2V1vQ62}{O padrão Bitcoin} do Saifedean Ammous;
\item \href{https://amzn.to/3jwpiG7}{Criptoativos} do Chris Burniske e Jack;
\item \href{https://tinyurl.com/bzrkbb5u}{Pesquisar no Google: Economia Austríaca};
\item \href{https://tinyurl.com/3yn8amt3}{Bitcoin Investment Theses} do Pierre Rochard;
\item \href{https://tinyurl.com/f8e4wn5h}{The Bullish Case for Bitcoin} do Vijay Boyapati;
\item \href{https://tinyurl.com/y9mhw2ad}{For kids: Bitcoin Money} do michael Caras.
\end{itemize}

Para se aprofundar na ciência da computação:

\begin{itemize}
\item O \href{https://bitcoin.org/bitcoin.pdf}{whitepaper do Bitcoin} escrito por Satoshi Nakamoto;
\item \href{https://amzn.to/3gOqaUH}{Mastering Bitcoin} do Andreas Antonopulous;
\item O \href{https://programmingbitcoin.com/#programming-blockchain}{Seminário} do Jimmy Song;
\item \href{https://programmingblockchain.gitbook.io/programmingblockchain/}{Programming Blockchain} também do Jimmy Song.
\end{itemize}

Para se aprofundar na história e filosofia do Bitcoin:

\begin{itemize}
\item \href{https://tinyurl.com/Planting-Bitcoin}{Planting Bitcoin} por Dan Held;
\item \href{https://tinyurl.com/mzy6jkz4}{Bitcoin Governance} do Pierre Richard;
\item \href{https://tinyurl.com/y8vueb88}{Bitcoin Past and Future} do Murad Mahmudov;
\item Todos os vídeos feitos por Andreas Antonopulous, especialmente \href{https://www.youtube.com/user/aantonop}{Currency Wars e The Monument of Immutability}.
\end{itemize}

Uma grande parte do ecossistema Bitcoin vive no Twitter. Deixarei aqui um punhado de pessoas, sem uma ordem específica, que seria interessante seguir. Comece nesta lista e vá diversificando conforme for encontrando novas mentes que caíram na toca do coelho:

\begin{itemize}
\item \href{https://twitter.com/lopp}{@lopp}
\item \href{https://twitter.com/pwuille}{@pwuille}
\item \href{https://twitter.com/adam3us}{@adam3us}
\item \href{https://twitter.com/danheld}{@danheld}
\item \href{https://twitter.com/TraceMayer}{@TraceMayer}
\item \href{https://twitter.com/pierre_rochard}{@pierre\_rochard}
\item \href{https://twitter.com/bitstein}{@bitstein}
\item \href{https://twitter.com/Melt_Dem}{@Melt\_Dem}
\item \href{https://twitter.com/theonevortex}{@theonevortex}
\item \href{https://twitter.com/WhatBitcoinDid}{@WhatBitcoinDid}
\item \href{https://twitter.com/stephanlivera}{@stephanlivera}
\item \href{https://twitter.com/TheBlock__}{@TheBlock\_\_}
\item \href{https://twitter.com/TheLTBNetwork}{@TheLTBNetwork}
\item \href{https://twitter.com/real_vijay}{@real\_vijay}
\item \href{https://twitter.com/jimmysong}{@jimmysong}
\item \href{https://twitter.com/Excellion}{@Excellion}
\item \href{https://twitter.com/starkness}{@starkness}
\item \href{https://twitter.com/roasbeef}{@roasbeef}
\item \href{https://twitter.com/saifedean}{@saifedean}
\item \href{https://twitter.com/giacomozucco}{@giacomozucco}
\item \href{https://twitter.com/Snyke}{@Snyke}
\item \href{https://twitter.com/aantonop}{@aantonop}
\item \href{https://twitter.com/MustStopMurad}{@MustStopMurad}
\item \href{https://twitter.com/peterktodd}{@peterktodd}
\item \href{https://twitter.com/skwp}{@skwp} (Autor do livro)
\item \href{https://twitter.com/KoreaComK}{@KoreaComK} (Tradutor do livro)
\end{itemize}

Você pode encontrar mais textos do autor do livro em\newline \url{https://yanpritzker.com}. Vejo você do outro lado.